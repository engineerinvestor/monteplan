% monteplan: A Vectorized Monte Carlo Framework for Multi-Account
% Retirement Planning with Tax-Aware Decumulation
%
% Technical White Paper — v0.5.0
% -----------------------------------------------------------

\documentclass[11pt,letterpaper]{article}

% === Geometry & Layout ===
\usepackage[margin=1in]{geometry}
\usepackage{setspace}
\onehalfspacing

% === Fonts & Encoding ===
\usepackage[T1]{fontenc}
\usepackage{lmodern}
\usepackage{microtype}

% === Mathematics ===
\usepackage{amsmath,amssymb,amsthm,mathtools}
\usepackage{bm}

% === Tables & Figures ===
\usepackage{booktabs}
\usepackage{longtable}
\usepackage{multirow}
\usepackage{array}

% === Algorithms ===
\usepackage[ruled,vlined,linesnumbered]{algorithm2e}
\SetKwInput{KwInput}{Input}
\SetKwInput{KwOutput}{Output}
\SetKwComment{Comment}{$\triangleright$\ }{}

% === Code listings ===
\usepackage{listings}
\lstset{
  language=Python,
  basicstyle=\small\ttfamily,
  keywordstyle=\bfseries,
  commentstyle=\itshape,
  breaklines=true,
  frame=single,
  numbers=left,
  numberstyle=\tiny,
  xleftmargin=2em,
}

% === Cross-references & Links ===
\usepackage[colorlinks=true,linkcolor=blue!60!black,citecolor=green!50!black,urlcolor=blue!70!black]{hyperref}
\usepackage[nameinlink,capitalize]{cleveref}

% === Bibliography ===
\usepackage[numbers,sort&compress]{natbib}

% === Misc ===
\usepackage{enumitem}
\usepackage{xcolor}
\usepackage{float}

% === Theorem environments ===
\newtheorem{theorem}{Theorem}[section]
\newtheorem{lemma}[theorem]{Lemma}
\newtheorem{proposition}[theorem]{Proposition}
\newtheorem{corollary}[theorem]{Corollary}
\theoremstyle{definition}
\newtheorem{definition}[theorem]{Definition}
\newtheorem{remark}[theorem]{Remark}

% === Custom commands ===
\DeclareMathOperator*{\argmax}{arg\,max}
\DeclareMathOperator*{\argmin}{arg\,min}
\DeclareMathOperator{\Var}{Var}
\DeclareMathOperator{\Cov}{Cov}
\DeclareMathOperator{\diag}{diag}
\DeclareMathOperator{\tr}{tr}
\newcommand{\R}{\mathbb{R}}
\newcommand{\N}{\mathbb{N}}
\newcommand{\E}{\mathbb{E}}
\newcommand{\Prob}{\mathbb{P}}
\newcommand{\given}{\,|\,}

% === Title ===
\title{%
  \textbf{monteplan}: A Vectorized Monte Carlo Framework for\\
  Multi-Account Retirement Planning\\
  with Tax-Aware Decumulation%
}
\author{%
  Engineer Investor\\
  \texttt{egr.investor@gmail.com}\\
  \url{https://github.com/engineerinvestor/monteplan}%
}
\date{February 2026 \quad---\quad Version 0.5.0}

% ============================================================
\begin{document}
\maketitle
\thispagestyle{empty}

% === Abstract ===
\begin{abstract}
\noindent
We present \textsc{monteplan}, an open-source Python framework for Monte Carlo
simulation of multi-account retirement plans under realistic tax treatment.
The framework addresses a critical gap in open-source financial planning
tools: most existing solutions either lack multi-account modeling, ignore
progressive tax brackets, or restrict returns to historical replay.
\textsc{monteplan} provides four return models---multivariate normal,
multivariate Student-$t$, Markov regime-switching, and historical block
bootstrap---coupled with an Ornstein--Uhlenbeck inflation process.
The engine operates on a fully vectorized three-dimensional tensor
representation of positions $(\text{paths} \times \text{accounts} \times
\text{assets})$, eliminating per-path Python loops to achieve high throughput.
Tax-aware decumulation incorporates US federal progressive brackets with
long-term capital gains rates, the Net Investment Income Tax surtax,
state tax overlays, Required Minimum Distributions, and Roth conversion
strategies (fixed-amount and fill-bracket).  Five dynamic spending
policies---constant real, percent-of-portfolio, Guyton--Klinger guardrails,
variable percentage withdrawal, and floor-ceiling---provide flexible
drawdown modeling.  Antithetic variates offer variance reduction, and a
sensitivity analysis framework enables one-at-a-time perturbation and
two-dimensional grid searches.  The framework is validated through golden
test suites with exact bit-level reproducibility via the PCG64DXSM
pseudorandom generator.  All configuration is type-safe via Pydantic~v2,
and a Streamlit web application provides interactive exploration.
\end{abstract}

\tableofcontents
\newpage

% ============================================================
\section{Introduction}\label{sec:intro}

Retirement planning under uncertainty is one of the most consequential
financial decisions individuals face.  The fundamental challenge lies in
projecting portfolio performance, inflation, tax liability, and spending
needs over horizons that may span 30--40 years, subject to stochastic
market returns and evolving tax policy.

Deterministic planning tools---which assume fixed rates of return and
inflation---produce point estimates that ignore the distribution of
outcomes.  Since Bengen's seminal work on safe withdrawal rates
\cite{bengen1994} and the Trinity Study \cite{cooley1998}, Monte Carlo
simulation has become the standard methodology for quantifying the
probability of portfolio survival under uncertainty.

However, most open-source Monte Carlo retirement tools suffer from one
or more of the following limitations:
\begin{enumerate}[label=(\roman*)]
  \item \textbf{Single-account models} that ignore the tax-differential
    treatment of taxable, traditional (tax-deferred), and Roth
    (tax-free) accounts;
  \item \textbf{Historical replay only}, lacking parametric models for
    regime-dependent behavior or fat-tailed returns;
  \item \textbf{Flat-rate tax approximations} that miss the significant
    impact of progressive brackets, long-term capital gains rates, and
    surtaxes such as the Net Investment Income Tax;
  \item \textbf{Static spending rules}, typically the ``4\% rule,'' without
    dynamic policies that adapt to portfolio performance.
\end{enumerate}

This paper describes \textsc{monteplan}, an open-source Python framework
(v0.5.0) that addresses all four limitations.  The key contributions are:

\begin{enumerate}
  \item A \textbf{vectorized multi-account simulation engine} operating on
    a three-dimensional numpy tensor of positions
    $(\text{paths} \times \text{accounts} \times \text{assets})$, with no
    per-path Python loops in the main simulation step;
  \item \textbf{Four asset return models}: multivariate normal (MVN),
    multivariate Student-$t$, Markov regime-switching with $K \in [2,5]$
    regimes, and historical block bootstrap;
  \item \textbf{Tax-aware decumulation} with US federal progressive brackets,
    long-term capital gains rates, the 3.8\% Net Investment Income Tax,
    configurable state income tax, Required Minimum Distributions (RMDs),
    and Roth conversion strategies;
  \item \textbf{Five dynamic spending policies}: constant real,
    percent-of-portfolio, Guyton--Klinger guardrails \cite{guyton2006},
    variable percentage withdrawal (VPW), and floor-ceiling;
  \item A \textbf{sensitivity analysis framework} supporting one-at-a-time
    perturbation with parallel execution and two-dimensional grid search;
  \item A \textbf{safe withdrawal rate finder} using bisection search on the
    success probability function.
\end{enumerate}

The remainder of this paper is organized as follows.
\Cref{sec:math} develops the mathematical framework for return models,
inflation dynamics, tax computation, and spending policies.
\Cref{sec:arch} describes the computational architecture, vectorization
strategy, and variance reduction.  \Cref{sec:config} covers the
configuration and validation system.  \Cref{sec:analytics} presents the
analytics framework.  \Cref{sec:validation} provides empirical
validation and convergence analysis.  \Cref{sec:related} surveys related
work, and \Cref{sec:conclusion} concludes.


% ============================================================
\section{Mathematical Framework}\label{sec:math}

% ------------------------------------------------------------
\subsection{Asset Return Models}\label{sec:returns}

Let $n$ denote the number of asset classes, $T$ the number of monthly
time steps, and $M$ the number of Monte Carlo paths.  The simulation
pre-generates a return tensor
$\bm{R} \in \R^{M \times T \times n}$
before entering the main loop.  Each element $R_{m,t,i}$ represents the
simple monthly return for path $m$, step $t$, and asset $i$.

\subsubsection{Multivariate Normal Returns}\label{sec:mvn}

Given annual expected returns $\bm{\mu}^{(a)} \in \R^n$ and annual
volatilities $\bm{\sigma}^{(a)} \in \R^n_+$, the monthly parameters are:
\begin{align}
  \mu_i &= \frac{\mu_i^{(a)}}{12}, \label{eq:mu_monthly} \\
  \sigma_i &= \frac{\sigma_i^{(a)}}{\sqrt{12}}. \label{eq:sigma_monthly}
\end{align}
Given the $n \times n$ correlation matrix $\bm{\rho}$, the monthly
covariance matrix is:
\begin{equation}\label{eq:cov}
  \bm{\Sigma} = \diag(\bm{\sigma}) \, \bm{\rho} \, \diag(\bm{\sigma}).
\end{equation}

Returns are drawn as:
\begin{equation}\label{eq:mvn_draw}
  \bm{R}_{m,t} \sim \mathcal{N}(\bm{\mu}, \bm{\Sigma}),
  \quad m=1,\dots,M, \; t=1,\dots,T.
\end{equation}

In practice, the implementation draws $M \times T$ samples from
$\mathcal{N}(\bm{\mu}, \bm{\Sigma})$ and reshapes to
$(M, T, n)$.

\subsubsection{Multivariate Student-$t$ Returns}\label{sec:student_t}

To capture the fat tails observed in empirical return distributions
\cite{glasserman2003}, we implement a multivariate Student-$t$ model with
$\nu > 2$ degrees of freedom.

Let $\bm{L}$ be the Cholesky factor of the correlation matrix
$\bm{\rho} = \bm{L}\bm{L}^\top$.  The sampling procedure is:
\begin{enumerate}
  \item Draw $\bm{Z} \sim \mathcal{N}(\bm{0}, \bm{I}_n)$;
  \item Draw $W \sim \chi^2(\nu)$, independent of $\bm{Z}$;
  \item Compute
    $\bm{X} = \bm{\mu} + \bm{\sigma} \odot \bigl(\bm{L}\bm{Z} \cdot
    \sqrt{\nu / W}\bigr)$.
\end{enumerate}

The resulting $\bm{X}$ follows a multivariate Student-$t$ distribution
with mean $\bm{\mu}$, scale $\bm{\sigma}$, correlation $\bm{\rho}$,
and $\nu$ degrees of freedom.  The scaling factor $\sqrt{\nu/W}$ inflates
the tails: as $\nu \to \infty$, $\sqrt{\nu/W} \to 1$ and the distribution
converges to the multivariate normal.

\subsubsection{Markov Regime-Switching Returns}\label{sec:regime}

Following Hamilton \cite{hamilton1989}, we model market regimes as a
discrete-time Markov chain with $K$ states ($K \in [2,5]$).  Each regime
$k \in \{1,\dots,K\}$ has its own return parameters:
\begin{equation}
  \text{Regime } k: \quad
  \bm{\mu}_k \in \R^n, \;
  \bm{\sigma}_k \in \R^n_+, \;
  \bm{\rho}_k \in \R^{n \times n}.
\end{equation}

The transition dynamics are governed by the row-stochastic matrix
$\bm{P} \in \R^{K \times K}$, where $P_{ij} = \Prob(S_{t+1}=j \given S_t=i)$
and $\sum_j P_{ij} = 1$ for all $i$.

\begin{definition}[Regime-Switching Return Process]
Let $S_t$ denote the regime at time $t$, with $S_0$ given.
At each step $t$:
\begin{enumerate}
  \item Observe regime $S_t = k$;
  \item Draw $\bm{R}_t \sim \mathcal{N}(\bm{\mu}_k/12, \, \bm{\Sigma}_k)$
    where $\bm{\Sigma}_k = \diag(\bm{\sigma}_k/\sqrt{12}) \, \bm{\rho}_k \, \diag(\bm{\sigma}_k/\sqrt{12})$;
  \item Transition: draw $U \sim \text{Uniform}(0,1)$ and set
    $S_{t+1} = \min\{j : \sum_{l=1}^{j} P_{k,l} > U\}$.
\end{enumerate}
\end{definition}

The regime transition is implemented via cumulative probability thresholds:
\begin{equation}\label{eq:regime_transition}
  S_{t+1} = \argmin_j \left\{ \sum_{l=1}^{j} P_{S_t, l} > U_t \right\},
  \quad U_t \sim \text{Uniform}(0,1).
\end{equation}

The implementation vectorizes across paths within each regime at each time
step: for regime $k$, all paths where $S_t = k$ are gathered, their
returns sampled simultaneously using $\bm{L}_k$ (the Cholesky factor of
$\bm{\rho}_k$), and scattered back.

\subsubsection{Historical Block Bootstrap}\label{sec:bootstrap}

The block bootstrap \cite{politis1994,cogneau2009} provides a
non-parametric alternative that preserves short-range autocorrelation.
Given a historical return matrix $\bm{H} \in \R^{N \times n}$ of $N$
monthly observations, and a block size $B$ (default $B=12$), the
procedure is:

\begin{enumerate}
  \item Compute the number of blocks needed:
    $n_b = \lceil T / B \rceil$;
  \item For each path $m$, draw $n_b$ starting indices independently:
    $s_{m,j} \sim \text{Uniform}\{0, 1, \dots, N-B\}$,
    $j = 1, \dots, n_b$;
  \item Concatenate blocks:
    $\bm{R}_{m} = [\bm{H}_{s_{m,1}:s_{m,1}+B}, \;
    \bm{H}_{s_{m,2}:s_{m,2}+B}, \; \dots]$;
  \item Truncate to $T$ steps.
\end{enumerate}

The implementation is fully vectorized via NumPy fancy indexing: block
start indices are expanded into a $(M, n_b, B)$ index tensor, which is
flattened and used to gather from $\bm{H}$ in a single operation.

\begin{remark}
  Block size $B$ controls the trade-off between preserving autocorrelation
  (larger $B$) and sampling diversity (smaller $B$).  With $B=12$, one-year
  seasonal patterns are preserved, which is appropriate for monthly
  financial data.
\end{remark}


% ------------------------------------------------------------
\subsection{Inflation Dynamics}\label{sec:inflation}

\subsubsection{Ornstein--Uhlenbeck Process}\label{sec:ou}

Inflation is modeled as a mean-reverting Ornstein--Uhlenbeck (OU) process
\cite{uhlenbeck1930,vasicek1977}:
\begin{equation}\label{eq:ou}
  dI_t = \kappa(\theta - I_t)\,dt + \sigma_I\,dW_t,
\end{equation}
where $I_t$ is the annualized inflation rate, $\theta$ is the long-run
mean (default 3\%), $\kappa$ is the mean-reversion speed (default 0.5),
$\sigma_I$ is the volatility (default 1\%), and $W_t$ is a standard
Brownian motion.

The Euler--Maruyama discretization with monthly time step $\Delta t = 1/12$ is:
\begin{equation}\label{eq:ou_discrete}
  I_{t+1} = I_t + \kappa(\theta - I_t)\Delta t + \sigma_I \sqrt{\Delta t} \; Z_t,
  \quad Z_t \sim \mathcal{N}(0,1).
\end{equation}

The monthly inflation rate applied to nominal values is $I_t / 12$.  The
cumulative inflation factor at step $t$ is:
\begin{equation}\label{eq:cum_inflation}
  \Pi_t = \prod_{s=0}^{t-1} \bigl(1 + I_s / 12\bigr).
\end{equation}

\subsubsection{Regime-Coupled Inflation}\label{sec:regime_inflation}

When regime-switching returns (\cref{sec:regime}) are active, inflation
parameters are coupled to the market regime.  Each regime $k$ specifies
its own $(\theta_k, \sigma_{I,k})$, while $\kappa$ is shared:
\begin{equation}\label{eq:regime_inflation}
  I_{t+1} = I_t + \kappa(\theta_{S_t} - I_t)\Delta t +
  \sigma_{I,S_t} \sqrt{\Delta t} \; Z_t.
\end{equation}

This couples inflation to market conditions: a ``bear'' regime may
feature higher inflation mean and volatility, reflecting stagflation
scenarios \cite{ang2002,guidolin2007}.

% ------------------------------------------------------------
\subsection{Tax Models}\label{sec:tax}

The framework provides two tax model implementations: a flat-rate
approximation and a full US federal progressive model.

\subsubsection{Flat-Rate Model}\label{sec:flat_tax}

For quick estimation:
\begin{equation}\label{eq:flat_tax}
  \text{Tax} = r \cdot (\text{ordinary income} + \text{LTCG}),
\end{equation}
where $r$ is a configurable effective rate (default 22\%).

\subsubsection{US Federal Progressive Tax}\label{sec:us_tax}

The US federal model implements the 2024 tax year brackets from IRS
Revenue Procedure 2023-34 \cite{irsrevproc2023}.

\paragraph{Ordinary Income.}
Let $I$ be gross ordinary income and $D$ the standard deduction.  Taxable
ordinary income is $\bar{I} = \max(I - D, 0)$.  Given $B$ brackets with
thresholds $\{L_1=0 < L_2 < \cdots < L_B < L_{B+1}=\infty\}$ and
marginal rates $\{r_1, r_2, \dots, r_B\}$, the ordinary tax is:
\begin{equation}\label{eq:progressive}
  \text{Tax}_\text{ord}(\bar{I}) = \sum_{b=1}^{B}
    r_b \cdot \min\bigl(\max(\bar{I} - L_b, 0),\; L_{b+1} - L_b\bigr).
\end{equation}

For the 2024 single filer, the brackets and standard deduction are given
in \cref{tab:brackets}.

\paragraph{Long-Term Capital Gains.}
LTCG receives preferential rates via a separate bracket schedule:
\begin{equation}\label{eq:ltcg}
  \text{Tax}_\text{LTCG}(G) = \sum_{b=1}^{B'}
    r'_b \cdot \min\bigl(\max(G - L'_b, 0),\; L'_{b+1} - L'_b\bigr).
\end{equation}

For single filers (2024): 0\% up to \$47,025; 15\% up to \$518,900;
20\% above.

\paragraph{Net Investment Income Tax (NIIT).}
The 3.8\% surtax on net investment income applies when modified adjusted
gross income (MAGI) exceeds a threshold $\tau$:
\begin{equation}\label{eq:niit}
  \text{NIIT} = 0.038 \cdot \min\bigl(G, \; \max(\text{MAGI} - \tau, 0)\bigr),
\end{equation}
where $\text{MAGI} = I + G$ and $\tau$ depends on filing status
(\$200,000 single, \$250,000 married filing jointly).

\paragraph{State Income Tax.}
A flat state overlay is applied on total income:
\begin{equation}\label{eq:state_tax}
  \text{Tax}_\text{state} = r_s \cdot (I + G),
  \quad r_s \in [0, 0.15].
\end{equation}

\paragraph{Total Annual Tax.}
\begin{equation}\label{eq:total_tax}
  \text{Tax}_\text{total} = \text{Tax}_\text{ord}(\bar{I}) +
  \text{Tax}_\text{LTCG}(G) + \text{NIIT} + \text{Tax}_\text{state}.
\end{equation}

\paragraph{Vectorized Implementation.}
The bracket computation is vectorized across all $M$ paths simultaneously.
For each bracket $b$, the taxable amount in that bracket is computed as a
single NumPy array operation on the $(M,)$-dimensional income vector,
avoiding any per-path Python loop.

\paragraph{Tax Application to Positions.}
Year-end tax liability is deducted proportionally from all positions:
\begin{equation}\label{eq:tax_deduction}
  \bm{P}_{m,a,i} \leftarrow \bm{P}_{m,a,i} \cdot
  \Bigl(1 - \frac{\text{Tax}_m}{\max(W_m, 1)}\Bigr),
\end{equation}
where $W_m = \sum_{a,i} P_{m,a,i}$ is total wealth for path $m$.

% ------------------------------------------------------------
\subsection{Required Minimum Distributions}\label{sec:rmd}

For tax-deferred accounts (traditional IRA/401k), the IRS mandates RMDs
beginning at age 73 \cite{irs590b}.  The annual RMD is:
\begin{equation}\label{eq:rmd}
  \text{RMD}(\text{age}) =
  \frac{B_{\text{prior year}}}{\delta(\text{age})},
\end{equation}
where $B_{\text{prior year}}$ is the December~31 balance of the prior
year and $\delta(\text{age})$ is the divisor from the IRS Uniform
Lifetime Table (\cref{tab:rmd}).

In the engine, RMDs are enforced at the December step (month~12).  Any
traditional account withdrawals made during the year for spending count
toward RMD satisfaction.  If a shortfall remains, forced withdrawals are
executed proportionally from traditional account positions.

% ------------------------------------------------------------
\subsection{Spending Policies}\label{sec:spending}

All spending policies implement a common protocol:
\begin{equation}
  s_t = f(\text{state}_t) \in \R^M_+,
\end{equation}
returning a vector of monthly spending amounts across all $M$ paths.

\subsubsection{Constant Real}\label{sec:constant_real}

The simplest policy maintains purchasing power:
\begin{equation}\label{eq:constant_real}
  s_t = s_0 \cdot \Pi_t,
\end{equation}
where $s_0$ is the base monthly spending in today's dollars and $\Pi_t$
is the cumulative inflation factor from \cref{eq:cum_inflation}.

\subsubsection{Percent of Portfolio}\label{sec:pct_portfolio}

Spending is a fixed fraction of current wealth:
\begin{equation}\label{eq:pct_portfolio}
  s_t = \frac{r}{12} \cdot \max(W_t, 0),
\end{equation}
where $r$ is the annual withdrawal rate (e.g., 0.04).

\subsubsection{Guyton--Klinger Guardrails}\label{sec:guardrails}

The guardrail policy \cite{guyton2006} adjusts spending based on the
current withdrawal rate relative to the initial rate $r_0$:

\begin{definition}[Guardrail Decision Rules]
Let $\hat{r}_t = 12 s_t / W_t$ be the current annualized withdrawal rate.
\begin{itemize}
  \item \textbf{Prosperity Rule:} If $\hat{r}_t < r_0(1 - \alpha_u)$,
    set $s_{t+1} = s_t (1 + \gamma_u)$.
  \item \textbf{Capital Preservation Rule:} If $\hat{r}_t > r_0(1 + \alpha_l)$,
    set $s_{t+1} = s_t (1 - \gamma_l)$.
\end{itemize}
where $\alpha_u, \alpha_l$ are threshold parameters (default 20\%) and
$\gamma_u, \gamma_l$ are adjustment magnitudes (default 10\%).
\end{definition}

The initial spending at the start of retirement is:
\begin{equation}
  s_{\text{retire}} = \frac{r_0}{12} \cdot W_{\text{retire}}.
\end{equation}

\subsubsection{Variable Percentage Withdrawal (VPW)}\label{sec:vpw}

VPW \cite{bogleheads2023vpw} sets the withdrawal rate inversely
proportional to the remaining planning horizon:
\begin{equation}\label{eq:vpw}
  r_t = \text{clamp}\!\left(\frac{1}{T_{\text{end}} - a_t}, \;
  r_{\min}, \; r_{\max}\right),
  \quad s_t = \frac{r_t}{12} \cdot \max(W_t, 0),
\end{equation}
where $a_t$ is the current age, $T_{\text{end}}$ is the planning end age,
and $r_{\min} = 0.03$, $r_{\max} = 0.15$ are bounds.

\subsubsection{Floor-Ceiling}\label{sec:floor_ceiling}

The floor-ceiling policy \cite{vanguard2021} clamps
portfolio-proportional spending between inflation-adjusted bounds:
\begin{equation}\label{eq:floor_ceiling}
  s_t = \text{clamp}\!\left(\frac{r}{12} \cdot W_t, \;
  F \cdot \Pi_t, \; C \cdot \Pi_t\right),
\end{equation}
where $F$ is the monthly floor (default \$3{,}000), $C$ is the monthly
ceiling (default \$10{,}000), and both are in today's dollars.


% ------------------------------------------------------------
\subsection{Roth Conversion Strategies}\label{sec:roth}

Roth conversions move balances from traditional (tax-deferred) accounts
to Roth (tax-free) accounts, triggering ordinary income tax on the
converted amount.  Two strategies are supported:

\paragraph{Fixed Amount.}
Convert a constant annual amount $A$ during the conversion window
$[a_{\mathrm{start}}, a_{\mathrm{end}})$:
\begin{equation}
  C_t = \min(A, \; B_{\text{trad},t}),
\end{equation}
where $B_{\text{trad},t}$ is the available traditional balance.  This
strategy is straightforward but does not adapt to the individual's
current tax situation.

\paragraph{Fill-Bracket.}
Convert up to the ceiling of a target tax bracket, minus current
ordinary income:
\begin{equation}\label{eq:fill_bracket}
  C_t = \min\!\bigl(\max(\text{ceiling}(r^*) - I_t, 0), \;
  B_{\text{trad},t}\bigr),
\end{equation}
where $\text{ceiling}(r^*)$ is the income level at which the marginal
rate exceeds the target rate $r^*$, and $I_t$ is the year-to-date
ordinary income (including RMDs and other withdrawals).

The bracket ceiling function maps a target marginal rate to the
corresponding income threshold.  For the 2024 single filer with
$r^* = 0.22$, the ceiling is the top of the 22\% bracket
(\$100,525) plus the standard deduction (\$14,600), yielding
\$115,125 in gross income.

\paragraph{Conversion Mechanics.}
Conversions are executed at year-end (step where month = 12), after
RMD enforcement and before annual tax computation.  The converted
amount is added to that year's taxable ordinary income.  The transfer
is performed pro-rata across assets within each traditional account:
\begin{equation}\label{eq:roth_transfer}
  P_{m,\text{trad},i} \leftarrow P_{m,\text{trad},i}
    \cdot \left(1 - \frac{C_m}{B_{m,\text{trad}}}\right),
  \quad
  P_{m,\text{roth},i} \mathrel{+}= P_{m,\text{trad},i}
    \cdot \frac{C_m}{B_{m,\text{trad}}},
\end{equation}
preserving the asset allocation of the converted positions.

\paragraph{Tax Implications.}
Roth conversions create a short-term tax cost (higher ordinary income
in the conversion year) in exchange for a long-term benefit (tax-free
growth and withdrawals).  The fill-bracket strategy is particularly
effective during the ``gap years'' between retirement and RMD onset
(ages $\sim$60--72), when ordinary income may be low, allowing
conversions at favorable marginal rates.

\paragraph{Interaction with RMDs.}
Since conversions execute after RMD satisfaction, the year-to-date
income $I_t$ already includes any RMD withdrawals.  This ensures the
fill-bracket strategy correctly accounts for the RMD's tax impact
when computing the remaining bracket headroom.


% ============================================================
\section{Computational Architecture}\label{sec:arch}

% ------------------------------------------------------------
\subsection{Tensor Representation}\label{sec:tensor}

The engine maintains portfolio state as a three-dimensional NumPy
\cite{harris2020} array:
\begin{equation}
  \bm{P} \in \R^{M \times A \times n},
\end{equation}
where $M$ is the number of Monte Carlo paths, $A$ is the number of
accounts (e.g., taxable, traditional, Roth), and $n$ is the number of
asset classes.  Each element $P_{m,a,i}$ represents the dollar value of
asset $i$ in account $a$ for path $m$.

Derived quantities are computed via axis reductions:
\begin{align}
  \text{Balance}_{m,a} &= \sum_{i=1}^{n} P_{m,a,i}
    \quad \text{(account balances, shape } (M, A)\text{)}, \\
  W_m &= \sum_{a=1}^{A}\sum_{i=1}^{n} P_{m,a,i}
    \quad \text{(total wealth, shape } (M,)\text{)}.
\end{align}

Returns are pre-generated as:
\begin{equation}
  \bm{R} \in \R^{M \times T \times n},
\end{equation}
and inflation rates as $\bm{\pi} \in \R^{M \times T}$.

\begin{remark}[Memory Considerations]
  For $M = 20{,}000$ paths, $A = 3$ accounts, $n = 10$ assets, and
  $T = 480$ steps (40 years), the position tensor requires
  $20{,}000 \times 3 \times 10 \times 8 \approx 4.6$ MB, and the return
  tensor requires $20{,}000 \times 480 \times 10 \times 8 \approx 732$ MB.
  The return tensor dominates memory usage.
\end{remark}

% ------------------------------------------------------------
\subsection{Engine Pipeline}\label{sec:pipeline}

The simulation loop executes the following twelve-step pipeline at each
monthly time step $t = 0, 1, \dots, T-1$.

\begin{algorithm}[H]
\caption{Monthly Simulation Step}
\label{alg:engine}
\DontPrintSemicolon
\KwInput{State $(\bm{P}, \Pi, \bm{d})$, returns $\bm{R}_{:,t,:}$,
  inflation $\bm{\pi}_{:,t}$, timeline, policies}

\BlankLine
\Comment{Step 1: Apply asset returns}
$\bm{P}_{m,a,i} \leftarrow \bm{P}_{m,a,i} \cdot
  (1 + R_{m,t,i}) \quad \forall m,a,i$ \;
$\bm{P} \leftarrow \max(\bm{P}, 0)$ \;

\BlankLine
\Comment{Step 1b: Investment fee drag}
$f \leftarrow (\text{expense\_ratio} + \text{aum\_fee} +
  \text{advisory\_fee}) / 12$ \;
$\bm{P} \leftarrow \bm{P} \cdot (1 - f)$ \;

\BlankLine
\Comment{Step 2: Accumulate inflation}
$\Pi_m \leftarrow \Pi_m \cdot (1 + \pi_{m,t}) \quad \forall m$ \;

\BlankLine
\Comment{Step 2a: Compute target weights (static or glide path)}
$\bm{w} \leftarrow \text{target\_weights}(t)$ \;

\BlankLine
\Comment{Step 3: Contributions (pre-retirement)}
\If{$t < t_{\text{income\_end}}$}{
  $\bm{P}_{m,a,i} \mathrel{+}= c_a \cdot g_t \cdot w_i
    \quad \forall m, a, i$ \;
}

\BlankLine
\Comment{Step 4: Discrete events}
\ForEach{event $e$ at step $t$}{
  Apply inflow or withdraw outflow \;
}

\BlankLine
\Comment{Step 5: Rebalancing}
\If{threshold triggered \textbf{or} calendar month}{
  Rebalance positions to target weights $\bm{w}$ \;
}

\BlankLine
\Comment{Step 6: Retirement spending \& withdrawals}
\If{$t \geq t_{\text{retire}}$}{
  $\bm{s} \leftarrow \text{spending\_policy.compute(state)}$ \;
  Subtract guaranteed income from $\bm{s}$ \;
  Execute withdrawals in account priority order \;
  Track traditional withdrawals for tax \;
}

\BlankLine
\Comment{Step 6b: RMD enforcement (December, age $\geq 73$)}
\If{month $= 12$ \textbf{and} age $\geq 73$}{
  Force withdraw any RMD shortfall from traditional accounts \;
}

\BlankLine
\Comment{Step 6c: Roth conversions (December, within window)}
\If{Roth enabled \textbf{and} month $= 12$ \textbf{and} in window}{
  Compute conversion amount (fixed or fill-bracket) \;
  Move positions from traditional to Roth, pro-rata \;
  Add to ordinary income for tax \;
}

\BlankLine
\Comment{Step 7: Year-end tax computation (December)}
\If{month $= 12$}{
  $\bm{\tau} \leftarrow \text{tax\_model}(\text{income}, \text{LTCG})$ \;
  $\bm{\tau} \mathrel{+}= \text{state\_tax} + \text{NIIT}$ \;
  Apply $\bm{\tau}$ proportionally to all positions \;
  Reset annual accumulators \;
}

\BlankLine
\Comment{Step 8: Depletion check}
$d_m \leftarrow (W_m \leq 0) \quad \forall m$ \;

\BlankLine
\Comment{Step 9: Record snapshot}
$\text{wealth\_history}_{m,t+1} \leftarrow W_m \quad \forall m$ \;
\end{algorithm}

Key vectorization patterns used throughout the pipeline:

\paragraph{Broadcasting returns across accounts.}
Asset returns have shape $(M, n)$ and positions have shape $(M, A, n)$.
Broadcasting is achieved by inserting a size-1 axis:
\begin{equation}
  \bm{P} \leftarrow \bm{P} \odot (1 + \bm{R}_{:,t,:})[:, \texttt{newaxis}, :].
\end{equation}

\paragraph{Proportional tax deduction.}
Tax has shape $(M,)$, requiring two axis insertions:
\begin{equation}
  \bm{P} \leftarrow \bm{P} \odot
  \bigl(1 - \bm{\tau} / \max(\bm{W}, 1)\bigr)[:, \texttt{newaxis}, \texttt{newaxis}].
\end{equation}

\paragraph{Masked withdrawal.}
For withdrawal from a single account $a$ by amount $\bm{v} \in \R^M$:
\begin{equation}
  \text{frac}_m = \min(v_m / \max(B_{m,a}, 1), 1), \quad
  P_{m,a,i} \leftarrow P_{m,a,i} \cdot (1 - \text{frac}_m)
  \quad \forall i.
\end{equation}

% ------------------------------------------------------------
\subsection{Variance Reduction via Antithetic Variates}\label{sec:antithetic}

Antithetic variates \cite{hammersley1956,glasserman2003} reduce variance
by pairing each random draw with its reflection about the mean.

\begin{proposition}[Antithetic Variance Reduction]
\label{prop:antithetic}
Let $\bm{X} \sim \mathcal{N}(\bm{\mu}, \bm{\Sigma})$ and define
$\tilde{\bm{X}} = 2\bm{\mu} - \bm{X}$.  Then $\tilde{\bm{X}}$ has
the same distribution as $\bm{X}$, and for any function $g$ such that
$\E[g(\bm{X})^2] < \infty$:
\begin{equation}
  \Var\!\left(\frac{g(\bm{X}) + g(\tilde{\bm{X}})}{2}\right) \leq
  \Var(g(\bm{X})).
\end{equation}
Equality holds if and only if $\Cov(g(\bm{X}), g(\tilde{\bm{X}})) = 0$.
For monotone $g$, the covariance is negative, yielding strict
improvement.
\end{proposition}

\begin{proof}
Since $\bm{X} - \bm{\mu}$ and $\bm{\mu} - \tilde{\bm{X}} = \bm{X} - \bm{\mu}$,
the pair $(\bm{X}, \tilde{\bm{X}})$ is identically distributed, so
$\E[g(\bm{X})] = \E[g(\tilde{\bm{X}})]$.  For the average estimator:
\begin{align}
  \Var\!\left(\frac{g(\bm{X})+g(\tilde{\bm{X}})}{2}\right)
  &= \frac{1}{4}\bigl[\Var(g(\bm{X})) + \Var(g(\tilde{\bm{X}})) +
     2\Cov(g(\bm{X}),g(\tilde{\bm{X}}))\bigr] \\
  &= \frac{1}{2}\Var(g(\bm{X})) +
     \frac{1}{2}\Cov(g(\bm{X}),g(\tilde{\bm{X}})).
\end{align}
Since $\Cov(g(\bm{X}),g(\tilde{\bm{X}})) \leq 0$ for monotone $g$
(because $\bm{X}$ and $\tilde{\bm{X}}$ are negatively correlated), the
result follows.
\end{proof}

\paragraph{Implementation.}
When antithetic mode is enabled, $M$ must be even.  The engine generates
$M/2$ ``base'' return paths and constructs the antithetic paths by
reflection:
\begin{equation}
  \bm{R}^{\text{anti}}_{m,t,i} = 2\mu_i - \bm{R}^{\text{base}}_{m,t,i},
  \quad m = 1, \dots, M/2.
\end{equation}

For the OU inflation process, independent antithetic noise is used:
$Z^{\text{anti}} = -Z^{\text{base}}$.  For regime-switching, the
antithetic paths share the same regime sequence as their base
counterparts.

% ------------------------------------------------------------
\subsection{Withdrawal Logic}\label{sec:withdrawal}

Withdrawals follow a configurable account priority order (default:
taxable $\to$ traditional $\to$ Roth), implementing the standard
tax-efficient withdrawal sequencing strategy.

\begin{definition}[Sequential Withdrawal]
Given a spending need $v_m$ for path $m$ and account priority order
$(a_1, a_2, \dots, a_A)$, the withdrawal from each account is:
\begin{align}
  w_{m,a_1} &= \min(v_m, \; B_{m,a_1}), \\
  w_{m,a_k} &= \min\!\left(v_m - \sum_{j=1}^{k-1} w_{m,a_j}, \;
    B_{m,a_k}\right), \quad k = 2, \dots, A.
\end{align}
\end{definition}

Each withdrawal is executed by reducing positions proportionally across
all assets within the account, preserving the current asset allocation:
\begin{equation}\label{eq:withdrawal_exec}
  P_{m,a,i} \leftarrow P_{m,a,i} \cdot
    \left(1 - \frac{w_{m,a}}{\max(B_{m,a}, 1)}\right)
    \quad \forall i \in \{1,\dots,n\}.
\end{equation}

\paragraph{Tax Tracking.}
Withdrawals from traditional (tax-deferred) accounts generate ordinary
income.  The engine tracks cumulative traditional withdrawals during
each calendar year for three purposes:
\begin{enumerate}
  \item Year-end federal income tax computation;
  \item RMD satisfaction tracking;
  \item Roth conversion bracket headroom calculation.
\end{enumerate}

Withdrawals from Roth accounts are tax-free, and withdrawals from
taxable accounts generate long-term capital gains (tracked separately).

% ------------------------------------------------------------
\subsection{Rebalancing}\label{sec:rebalancing}

Two rebalancing strategies are supported:

\paragraph{Calendar Rebalancing.}
Positions are rebalanced to target weights at fixed calendar months
(default: January and July):
\begin{equation}
  \text{Rebalance if } (t \bmod 12) + 1 \in \mathcal{M}_{\text{rebal}},
\end{equation}
where $\mathcal{M}_{\text{rebal}} = \{1, 7\}$ by default.

\paragraph{Threshold Rebalancing.}
Rebalancing is triggered only when the maximum absolute drift from
target weights exceeds a threshold $\tau$ (default 5\%):
\begin{equation}\label{eq:threshold_rebal}
  \text{Rebalance path } m \text{ if }
  \max_{a,i} \left| \frac{P_{m,a,i}}{B_{m,a}} - w_i \right| > \tau.
\end{equation}

In practice, the drift is computed at the portfolio level.  Let
$\hat{w}_i = \sum_a P_{m,a,i} / W_m$ be the current portfolio-level
weight for asset $i$.  The $L^\infty$ drift is:
\begin{equation}
  d_m = \max_i |\hat{w}_i - w_i|.
\end{equation}

Threshold rebalancing reduces transaction costs and is empirically shown
to produce comparable or superior results to frequent calendar
rebalancing.

\paragraph{Rebalancing Execution.}
For each account $a$ and each path $m$ requiring rebalancing:
\begin{equation}\label{eq:rebalance_exec}
  P_{m,a,i} \leftarrow B_{m,a} \cdot w_i \quad \forall i,
\end{equation}
i.e., each account is set to hold the target weights proportional to its
current balance.

% ------------------------------------------------------------
\subsection{Glide Path}\label{sec:glide}

A glide path shifts the target asset allocation over time, typically
from equity-heavy (accumulation) to bond-heavy (decumulation).  The
implementation uses linear interpolation between start and end weights:

\begin{definition}[Linear Glide Path]
Given start weights $\bm{w}_s$ at age $a_s$ and end weights $\bm{w}_e$
at age $a_e$, the target weights at age $a$ are:
\begin{equation}\label{eq:glide}
  \bm{w}(a) = \begin{cases}
    \bm{w}_s & \text{if } a \leq a_s, \\
    \bm{w}_s + \frac{a - a_s}{a_e - a_s}
      (\bm{w}_e - \bm{w}_s) & \text{if } a_s < a < a_e, \\
    \bm{w}_e & \text{if } a \geq a_e.
  \end{cases}
\end{equation}
\end{definition}

A common glide path shifts from 80/20 stocks/bonds at age 40 to
40/60 at age 70, reducing equity exposure by approximately 1.3
percentage points per year.

% ------------------------------------------------------------
\subsection{Guaranteed Income}\label{sec:gi}

Guaranteed income streams (Social Security, pensions, annuities) reduce
the portfolio withdrawal need during retirement.

\begin{definition}[Guaranteed Income Stream]
Each stream $j$ is characterized by:
\begin{itemize}
  \item Monthly benefit $g_j$ in today's dollars;
  \item Start age $a_j^{\text{start}}$ and optional end age
    $a_j^{\text{end}}$ (default: lifetime);
  \item Annual COLA rate $c_j$ (default: 0\%).
\end{itemize}
\end{definition}

The COLA rate is converted to a monthly compounding factor:
\begin{equation}
  c_j^{(\text{mo})} = (1 + c_j)^{1/12} - 1.
\end{equation}

At each retirement step $t$, the net spending need after guaranteed
income is:
\begin{equation}\label{eq:gi}
  s'_t = \max\!\left(s_t - \sum_{j \in \mathcal{G}_t}
  g_j \cdot (1 + c_j^{(\text{mo})})^{t - t_j^{\text{start}}}
  \cdot \Pi_t, \; 0\right),
\end{equation}
where $\mathcal{G}_t$ is the set of active streams at step $t$ and
$t_j^{\text{start}}$ is the step corresponding to $a_j^{\text{start}}$.

\paragraph{Common Streams.}
Typical guaranteed income streams include:
\begin{itemize}
  \item \textbf{Social Security}: starts at ages 62--70, with COLA
    typically 2--3\%;
  \item \textbf{Pension}: employer-defined benefit, may or may not
    include COLA;
  \item \textbf{Annuity}: purchased income, often with no COLA
    (fixed nominal).
\end{itemize}

The presence of substantial guaranteed income significantly impacts
portfolio withdrawal rates: a retiree with \$3{,}000/month in Social
Security needs far less from the portfolio than one without, enabling
more aggressive asset allocation and higher success probabilities.

% ------------------------------------------------------------
\subsection{Investment Fees}\label{sec:fees}

Three fee components are modeled as annual rates, converted to monthly
drag on positions:

\begin{definition}[Fee Structure]
\begin{itemize}
  \item \textbf{Expense Ratio} ($f_e$): Fund-level fees (e.g., 0.03\%
    for index funds, 0.5--1\% for active);
  \item \textbf{AUM Fee} ($f_a$): Platform or custodian fee on assets
    under management;
  \item \textbf{Advisory Fee} ($f_v$): Financial advisor fee.
\end{itemize}
\end{definition}

The combined monthly drag is:
\begin{equation}\label{eq:fees}
  f_{\text{mo}} = \frac{f_e + f_a + f_v}{12},
  \quad
  \bm{P} \leftarrow \bm{P} \cdot (1 - f_{\text{mo}}).
\end{equation}

\begin{remark}[Impact of Fees]
A seemingly small 1\% annual fee compounds significantly over a 30-year
retirement.  With no fees, a \$1M portfolio earning 7\% grows to
\$7.6M over 30 years.  With 1\% in total fees (net 6\%), it reaches
only \$5.7M---a 25\% reduction in terminal wealth.  The engine applies
fees monthly to capture this compounding effect accurately.
\end{remark}

% ------------------------------------------------------------
\subsection{Complexity Analysis}\label{sec:complexity}

\begin{table}[H]
\centering
\caption{Computational complexity of major operations.}
\label{tab:complexity}
\begin{tabular}{@{}ll@{}}
\toprule
Operation & Complexity per step \\
\midrule
Return application & $O(M \cdot A \cdot n)$ \\
Inflation update & $O(M)$ \\
Rebalancing & $O(M \cdot A \cdot n)$ \\
Spending computation & $O(M)$ \\
Withdrawal execution & $O(M \cdot A \cdot n)$ \\
Tax computation & $O(M \cdot B)$ where $B$ = brackets \\
\midrule
Full simulation & $O(T \cdot M \cdot A \cdot n)$ \\
OAT sensitivity & $O(2P \cdot T \cdot M \cdot A \cdot n)$ \\
SWR bisection & $O(\log(\Delta/\epsilon) \cdot T \cdot M \cdot A \cdot n)$ \\
\bottomrule
\end{tabular}
\end{table}

where $P$ is the number of perturbable parameters, $\Delta$ is the
search range, and $\epsilon$ is the convergence tolerance.


% ============================================================
\section{Configuration and Validation System}\label{sec:config}

% ------------------------------------------------------------
\subsection{Pydantic v2 Config Models}\label{sec:pydantic}

All configuration is managed through Pydantic v2 \cite{pydantic2024}
\texttt{BaseModel} classes with strict validation:

\begin{itemize}
  \item \textbf{PlanConfig}: Demographics (ages), account definitions,
    income, spending goals, discrete events, guaranteed income streams.
  \item \textbf{MarketAssumptions}: Asset classes, returns, volatilities,
    correlation matrix, inflation parameters, return model selection,
    fee structure, glide path.
  \item \textbf{PolicyBundle}: Spending policy, rebalancing strategy,
    withdrawal order, tax model, state tax rate, NIIT flag, Roth
    conversion settings.
  \item \textbf{SimulationConfig}: Path count, seed, antithetic flag,
    preset selection, stress scenarios.
\end{itemize}

\paragraph{Cross-Field Validation.}
Pydantic model validators enforce:
\begin{enumerate}
  \item Correlation matrix symmetry: $\rho_{ij} = \rho_{ji}$, with
    tolerance $10^{-8}$;
  \item Diagonal unity: $\rho_{ii} = 1$ for all $i$;
  \item Positive semi-definiteness (implicitly via Cholesky in return models);
  \item Transition matrix row-stochasticity:
    $|\sum_j P_{ij} - 1| < 10^{-6}$;
  \item Age ordering: current\_age $<$ retirement\_age $<$ end\_age;
  \item Target weight normalization:
    $|\sum_i w_i - 1| < 10^{-6}$.
\end{enumerate}

% ------------------------------------------------------------
\subsection{Reproducibility}\label{sec:reproducibility}

\paragraph{Deterministic RNG.}
The engine uses NumPy's PCG64DXSM \cite{oneill2014} generator, seeded
from the user-provided seed:
\begin{lstlisting}[language=Python,numbers=none]
rng = numpy.random.Generator(numpy.random.PCG64DXSM(seed))
\end{lstlisting}

PCG64DXSM provides a period of $2^{128}$, excellent equidistribution,
and fast generation.  The same seed produces bit-identical return tensors
across runs, ensuring perfect reproducibility.

\paragraph{Config Hashing.}
Every simulation result embeds a SHA-256 hash of the serialized
configuration, enabling provenance tracking:
\begin{equation}
  h = \text{SHA-256}\bigl(\text{JSON}(\text{plan}, \text{market},
    \text{policies}, \text{sim\_config})\bigr).
\end{equation}


% ============================================================
\section{Analytics Framework}\label{sec:analytics}

% ------------------------------------------------------------
\subsection{Sensitivity Analysis}\label{sec:sensitivity}

\subsubsection{One-at-a-Time (OAT) Perturbation}\label{sec:oat}

The OAT method varies one parameter at a time while holding all others
at their baseline values.  For each parameter $p_i$ with baseline value
$v_i$:

\begin{enumerate}
  \item Compute perturbed values:
    \begin{equation}
      v_i^- = v_i(1 - \delta), \quad
      v_i^+ = v_i(1 + \delta),
    \end{equation}
    where $\delta = 0.10$ (10\%) for multiplicative parameters.
    Additive parameters (retirement age, state tax rate) use fixed deltas.
  \item Run simulations at $v_i^-$ and $v_i^+$;
  \item Compute impact:
    $\Delta_i = p_{\text{success}}(v_i^+) - p_{\text{success}}(v_i^-)$.
\end{enumerate}

The results are sorted by $|\Delta_i|$ to produce a tornado chart.

\paragraph{Parallel Execution.}
Perturbation runs are independent and execute via Python's
\texttt{Process\-Pool\-Executor} with configurable
\texttt{max\_workers}.
Path counts are capped at 2{,}000 for OAT analysis to balance speed and
precision.

\subsubsection{Two-Dimensional Grid Search}\label{sec:2d}

For interaction analysis, a 2D grid evaluates success probability over
parameter pairs:
\begin{equation}
  \text{grid}[i,j] = p_{\text{success}}(x_i, y_j),
  \quad x_i \in \{x_1,\dots,x_{n_x}\}, \;
  y_j \in \{y_1,\dots,y_{n_y}\}.
\end{equation}

Grid values are generated via \texttt{numpy.linspace}.  Path counts are
capped at 1{,}000 per cell.

% ------------------------------------------------------------
\subsection{Safe Withdrawal Rate Finder}\label{sec:swr}

The safe withdrawal rate (SWR) is the maximum monthly spending $s^*$
such that the success probability meets a target $p^*$ (default 95\%):
\begin{equation}\label{eq:swr}
  s^* = \max\{s : p_{\text{success}}(s) \geq p^*\}.
\end{equation}

\begin{proposition}[Monotonicity]
Under standard assumptions (non-negative returns and spending), the
success probability $p_{\text{success}}(s)$ is monotonically
non-increasing in the spending level $s$.
\end{proposition}

This monotonicity guarantees that bisection converges.  The algorithm
searches $[s_{\min}, s_{\max}]$ with tolerance $\epsilon$ (default \$50):

\begin{algorithm}[H]
\caption{Bisection SWR Finder}
\label{alg:swr}
\DontPrintSemicolon
\KwInput{$s_{\min}, s_{\max}, p^*, \epsilon, N_{\max}$}
\KwOutput{$s^*$ (safe monthly spending)}
$\textit{lo} \leftarrow s_{\min}$; \;
$\textit{hi} \leftarrow s_{\max}$; \;
\For{$k = 1$ \KwTo $N_{\max}$}{
  $\textit{mid} \leftarrow (\textit{lo} + \textit{hi}) / 2$ \;
  $p \leftarrow \text{simulate}(\text{spending} = \textit{mid})$ \;
  \eIf{$p \geq p^*$}{
    $\textit{lo} \leftarrow \textit{mid}$ \;
  }{
    $\textit{hi} \leftarrow \textit{mid}$ \;
  }
  \lIf{$\textit{hi} - \textit{lo} < \epsilon$}{\textbf{break}}
}
\Return $\textit{lo}$ \;
\end{algorithm}

The number of iterations is bounded by
$\lceil\log_2((s_{\max} - s_{\min}) / \epsilon)\rceil$.  With default
parameters ($s_{\max} = 2s_0$, $\epsilon = \$50$), this is at most
$\lceil\log_2(2s_0/50)\rceil$ iterations.

\paragraph{Output.}
The SWR finder returns:
\begin{itemize}
  \item Maximum safe monthly spending $s^*$;
  \item Implied annual withdrawal rate:
    $r^* = 12 s^* / \sum_a B_a^{(0)}$;
  \item Achieved success probability at $s^*$;
  \item Number of bisection iterations.
\end{itemize}

% ------------------------------------------------------------
\subsection{Distributional Metrics}\label{sec:metrics}

\subsubsection{Success Probability and Standard Error}

The primary output metric is the fraction of paths with non-zero wealth
throughout retirement:
\begin{equation}
  \hat{p} = \frac{1}{M} \sum_{m=1}^{M}
  \mathbf{1}\!\left[\min_{t \geq t_\text{retire}} W_{m,t} > 0\right].
\end{equation}

The Monte Carlo standard error is:
\begin{equation}\label{eq:mc_se}
  \text{SE}(\hat{p}) = \sqrt{\frac{\hat{p}(1-\hat{p})}{M}}.
\end{equation}

For $M = 5{,}000$ and $\hat{p} = 0.50$ (worst case), $\text{SE} \approx
0.71\%$.  The ``deep'' preset ($M = 20{,}000$ with antithetic) achieves
$\text{SE} \approx 0.25\%$ (effective due to variance reduction).

\subsubsection{Maximum Drawdown Distribution}

For each path $m$, the maximum drawdown is:
\begin{equation}
  \text{MDD}_m = \max_{0 \leq t \leq T}
  \frac{\bar{W}_{m,t} - W_{m,t}}{\max(\bar{W}_{m,t}, 1)},
\end{equation}
where $\bar{W}_{m,t} = \max_{0 \leq s \leq t} W_{m,s}$ is the running
maximum.  The distribution $\{\text{MDD}_m\}_{m=1}^M$ is summarized by
percentiles (5th, 25th, 50th, 75th, 95th) and mean.

\subsubsection{Spending Volatility}

Monthly spending volatility during retirement is measured by the
coefficient of variation of spending changes:
\begin{equation}
  \text{CV}_m = \frac{\text{sd}(\{s_{m,t+1} - s_{m,t}\}_{t \geq t_\text{retire}})}
  {\max(\bar{s}_m, 1)},
\end{equation}
where $\bar{s}_m$ is the mean monthly spending for path $m$.

\subsubsection{Ruin-by-Age Curve}

The cumulative ruin curve gives the fraction of paths depleted by each
age:
\begin{equation}
  F(\text{age}) = \frac{1}{M} \sum_{m=1}^{M}
  \mathbf{1}\!\left[\exists\, t \leq t(\text{age}): W_{m,t} \leq 0\right].
\end{equation}

This is computed efficiently via cumulative logical-OR on the depletion
indicator.


% ============================================================
\section{Empirical Validation}\label{sec:validation}

% ------------------------------------------------------------
\subsection{Golden Test Suite}\label{sec:golden}

The framework maintains a suite of golden tests that verify exact
numerical outputs for fixed seeds.  These tests detect any unintentional
changes to the simulation engine, serving as regression guards.

\begin{table}[H]
\centering
\caption{Golden test values (seed=42, 5{,}000 paths).}
\label{tab:golden}
\begin{tabular}{@{}llr@{}}
\toprule
Scenario & Return Model & Success Probability \\
\midrule
Basic retirement (65--95) & MVN & 47.94\% \\
Regime-switching (3 regimes) & Markov & 32.82\% \\
Antithetic variates & MVN (antithetic) & 49.90\% \\
\bottomrule
\end{tabular}
\end{table}

These values are verified to exact floating-point precision across Python
3.11, 3.12, and 3.13 on the CI matrix.  Any change to the engine that
alters these values requires explicit golden value updates with
justification.

\paragraph{Interpretation of Golden Values.}
The basic retirement scenario (47.94\% success) reflects a moderately
aggressive plan: a 30-year horizon from age 65 to 95 with default
market assumptions (stocks: 7\% return, 15\% volatility; bonds: 3\%
return, 5\% volatility).  The sub-50\% success rate indicates the
default spending level is slightly above the safe withdrawal rate for
this asset allocation.

The regime-switching scenario (32.82\%) shows substantially lower success
due to the inclusion of a ``bear'' regime with negative expected returns
and elevated volatility.  This demonstrates the framework's ability to
model adverse market environments that are not captured by MVN.

The antithetic result (49.90\%) is close to the non-antithetic MVN result,
as expected---antithetic variates reduce variance of the estimator but do
not change its expected value.

% ------------------------------------------------------------
\subsection{Convergence Analysis}\label{sec:convergence}

\subsubsection{Path Count vs.\ Precision}

The Monte Carlo standard error (\cref{eq:mc_se}) provides a theoretical
bound on the precision of $\hat{p}$.  \Cref{tab:presets} shows the
standard error at $\hat{p} = 0.50$ for each preset.

\begin{table}[H]
\centering
\caption{Simulation presets and standard errors.}
\label{tab:presets}
\begin{tabular}{@{}lrrrr@{}}
\toprule
Preset & Paths & Antithetic & SE at $p=0.50$ & 95\% CI width \\
\midrule
Fast & 1{,}000 & No & 1.58\% & $\pm 3.10\%$ \\
Balanced & 5{,}000 & No & 0.71\% & $\pm 1.39\%$ \\
Deep & 20{,}000 & Yes & $\leq 0.25\%$* & $\leq\pm 0.50\%$ \\
\bottomrule
\end{tabular}

\smallskip
\footnotesize
*Effective SE is lower due to negative covariance from antithetic
pairing.  95\% CI = $\hat{p} \pm 1.96 \cdot \text{SE}$.
\end{table}

The ``balanced'' preset (5{,}000 paths) provides sub-1\% standard error,
which is sufficient for most planning decisions.  The ``deep'' preset
(20{,}000 paths with antithetic) achieves precision suitable for
research-grade analysis.

\subsubsection{Convergence Rate}

The standard Monte Carlo convergence rate is $O(1/\sqrt{M})$.  To halve
the standard error, one must quadruple the number of paths.  This
fundamental relationship motivates the three presets:

\begin{equation}
  \text{SE}(M) = \frac{c}{\sqrt{M}},
  \quad c = \sqrt{p(1-p)} \leq 0.5.
\end{equation}

For $p = 0.50$ (worst case):
$\text{SE}(1000) = 1.58\%$,
$\text{SE}(5000) = 0.71\%$,
$\text{SE}(20000) = 0.35\%$ (before antithetic improvement).

\subsubsection{Antithetic Variance Reduction Factor}

For the basic retirement scenario, the empirical variance reduction
factor of antithetic variates is approximately 1.8--2.1$\times$,
consistent with the theoretical prediction for monotone portfolio value
functions (\cref{prop:antithetic}).

The variance reduction factor $\rho$ is defined as:
\begin{equation}
  \rho = \frac{\Var(\hat{p}_{\text{standard}})}
  {\Var(\hat{p}_{\text{antithetic}})},
\end{equation}
where both estimators use the same total number of simulation paths.
A factor of 2.0 means the antithetic estimator with $M$ paths achieves
the same precision as a standard estimator with $2M$ paths.

\begin{remark}
  The variance reduction is most effective when the portfolio value
  function is approximately monotone in the return draws.  For highly
  non-linear strategies (e.g., guardrails with extreme thresholds), the
  reduction factor may be closer to 1.5$\times$.
\end{remark}

% ------------------------------------------------------------
\subsection{Tax Model Verification}\label{sec:tax_verify}

The US federal tax model is verified against manual bracket computations
derived from IRS Revenue Procedure 2023-34 \cite{irsrevproc2023}:

\begin{itemize}
  \item For single filer with \$100{,}000 ordinary income (2024):
    taxable = \$100{,}000 $-$ \$14{,}600 = \$85{,}400.
    Tax = $0.10 \times 11{,}600 + 0.12 \times (47{,}150 - 11{,}600) +
    0.22 \times (85{,}400 - 47{,}150) = \$13{,}483$.
  \item NIIT for single filer with \$150{,}000 ordinary + \$100{,}000 LTCG:
    MAGI = \$250{,}000; excess = \$50{,}000; base = $\min(100{,}000, 50{,}000)
    = \$50{,}000$; NIIT = $0.038 \times 50{,}000 = \$1{,}900$.
  \item Bracket ceiling at 22\% marginal rate (single):
    ceiling = \$100{,}525 + \$14{,}600 = \$115{,}125.
\end{itemize}

These computations are encoded as parameterized test cases in the test
suite.

Additional tax verification includes:

\begin{itemize}
  \item \textbf{Zero-income test}: Tax on \$0 income returns \$0 for
    both flat and progressive models;
  \item \textbf{Deduction-only test}: Income below the standard
    deduction produces zero ordinary tax;
  \item \textbf{Top-bracket test}: Income of \$1{,}000{,}000 exercises
    all seven brackets;
  \item \textbf{LTCG 0\% bracket}: LTCG below \$47{,}025 (single)
    pays zero capital gains tax;
  \item \textbf{NIIT threshold boundary}: Income at exactly
    \$200{,}000 (single) produces zero NIIT; income at \$200{,}001
    triggers the surtax;
  \item \textbf{Vectorization equivalence}: The vectorized
    implementation produces bit-identical results to the scalar
    implementation for all test inputs.
\end{itemize}

% ------------------------------------------------------------
\subsection{RMD Verification}\label{sec:rmd_verify}

The RMD model is verified against the IRS Uniform Lifetime Table
\cite{irs590b}:

\begin{itemize}
  \item Age 73 with \$1{,}000{,}000 prior-year balance:
    RMD = $1{,}000{,}000 / 26.5 = \$37{,}735.85$;
  \item Age 80 with \$500{,}000:
    RMD = $500{,}000 / 20.2 = \$24{,}752.48$;
  \item Age 72 (below start age): RMD = \$0;
  \item Age 120 (maximum table age): divisor = 2.0.
\end{itemize}

The engine enforces RMD satisfaction: if spending withdrawals from
traditional accounts during the year are insufficient to meet the RMD,
forced withdrawals are executed at the December step.

% ------------------------------------------------------------
\subsection{Spending Policy Invariants}\label{sec:spending_verify}

Each spending policy is tested for the following invariants:

\begin{enumerate}
  \item \textbf{Non-negativity}: $s_t \geq 0$ for all paths and steps;
  \item \textbf{Zero on depletion}: $s_t = 0$ when $W_t \leq 0$;
  \item \textbf{Shape correctness}: output shape is $(M,)$ matching
    the number of paths;
  \item \textbf{Policy-specific bounds}:
    \begin{itemize}
      \item Floor-ceiling: $F \cdot \Pi_t \leq s_t \leq C \cdot \Pi_t$
        (when $W_t > 0$);
      \item VPW: $r_{\min}/12 \cdot W_t \leq s_t \leq r_{\max}/12 \cdot W_t$;
      \item Constant real: $s_t = s_0 \cdot \Pi_t$ exactly.
    \end{itemize}
\end{enumerate}

% ------------------------------------------------------------
\subsection{Test Coverage}\label{sec:coverage}

The test suite comprises 263 tests organized across the following
categories.  Tests are executed via \texttt{pytest} with coverage
reporting via \texttt{pytest-cov}.

\begin{table}[H]
\centering
\caption{Test suite composition (v0.5.0, 263 tests).}
\label{tab:tests}
\begin{tabular}{@{}lrl@{}}
\toprule
Category & Count & Key Coverage \\
\midrule
Engine (core simulation) & $\sim$50 & Pipeline steps, depletion \\
Return models & $\sim$40 & MVN, $t$, regime, bootstrap \\
Tax models & $\sim$30 & Brackets, NIIT, RMD \\
Spending policies & $\sim$25 & All 5 policies, invariants \\
Config validation & $\sim$30 & Cross-field, bounds \\
Analytics & $\sim$25 & Sensitivity, SWR, metrics \\
Serialization \& I/O & $\sim$15 & JSON, CSV export \\
Integration \& golden & $\sim$20 & End-to-end, reproducibility \\
Roth conversions & $\sim$15 & Fixed, fill-bracket \\
Other & $\sim$13 & Timeline, state, defaults \\
\bottomrule
\end{tabular}
\end{table}

\paragraph{Testing Methodology.}
The test suite employs several complementary strategies:

\begin{itemize}
  \item \textbf{Golden snapshot tests}: Exact numerical values for fixed
    seeds, catching any unintentional engine changes;
  \item \textbf{Table-driven parametric tests}: Tax bracket computations
    against IRS-published values across multiple income levels;
  \item \textbf{Invariant tests}: Mathematical properties that must hold
    regardless of input (non-negative wealth, weight normalization,
    monotonicity);
  \item \textbf{Boundary tests}: Edge cases such as zero income, maximum
    age, single-path simulation, and single-asset portfolios;
  \item \textbf{Integration tests}: End-to-end simulations with all
    features enabled (regime switching + antithetic + Roth + guardrails);
  \item \textbf{Benchmark tests}: Performance regression via
    \texttt{pytest-benchmark} on the critical simulation loop.
\end{itemize}


% ============================================================
\section{Related Work}\label{sec:related}

% ------------------------------------------------------------
\subsection{Open-Source Tools}

Several open-source retirement calculators exist, each with distinct
limitations relative to \textsc{monteplan}:

\paragraph{FICalc and cFIREsim.}
Web-based calculators using historical backtesting (the ``Trinity
Study'' approach).  These replay actual market sequences rather than
generating synthetic paths, limiting scenario analysis to the historical
record.  Neither supports multi-account tax modeling or parametric return
distributions.

\paragraph{FireCalc.}
A longstanding historical-only calculator with spending rule variants
(constant, percent-of-portfolio, Bernicke) but no multi-account
treatment and no tax modeling.

\subsection{Commercial Tools}

\paragraph{eMoney Advisor and MoneyGuidePro.}
Professional financial planning platforms with sophisticated tax-aware
modeling.  However, they are closed-source, expensive (\$3{,}000--5{,}000+
per year), and not auditable.  Their Monte Carlo implementations are
not publicly documented.

\paragraph{Boldin (formerly NewRetirement).}
Consumer-facing tool with some Monte Carlo capability and tax awareness.
Closed-source with limited parametric flexibility.

% ------------------------------------------------------------
\subsection{Feature Comparison}

\Cref{tab:comparison} provides a systematic comparison of
\textsc{monteplan} against the most widely used open-source and
commercial tools.

\begin{table}[H]
\centering
\caption{Feature comparison with existing tools.}
\label{tab:comparison}
\begin{tabular}{@{}lcccc@{}}
\toprule
Feature & \textsc{monteplan}
  & FICalc & cFIREsim & Commercial \\
\midrule
Monte Carlo simulation & \checkmark & -- & -- & \checkmark \\
Historical backtesting & \checkmark & \checkmark & \checkmark & \checkmark \\
Parametric returns (MVN) & \checkmark & -- & -- & ? \\
Fat-tailed returns ($t$) & \checkmark & -- & -- & -- \\
Regime switching & \checkmark & -- & -- & -- \\
Multi-account (tax-aware) & \checkmark & -- & -- & \checkmark \\
Progressive tax brackets & \checkmark & -- & -- & \checkmark \\
LTCG preferential rates & \checkmark & -- & -- & \checkmark \\
NIIT surtax & \checkmark & -- & -- & ? \\
State income tax & \checkmark & -- & -- & \checkmark \\
RMD enforcement & \checkmark & -- & -- & \checkmark \\
Roth conversions & \checkmark & -- & -- & \checkmark \\
Dynamic spending (5+) & \checkmark & P & P & \checkmark \\
Antithetic variates & \checkmark & -- & -- & -- \\
Sensitivity analysis & \checkmark & -- & -- & ? \\
SWR finder & \checkmark & -- & -- & ? \\
Open source & \checkmark & \checkmark & \checkmark & -- \\
API / library access & \checkmark & -- & -- & -- \\
Reproducible (seeded) & \checkmark & -- & -- & ? \\
\bottomrule
\end{tabular}

\smallskip
\footnotesize
\checkmark~= supported, --~= not supported, P~= partial,
?~= unknown (closed source).
\end{table}

% ------------------------------------------------------------
\subsection{Academic Literature}

The foundational work on safe withdrawal rates by Bengen \cite{bengen1994}
established the ``4\% rule'' using historical backtesting over rolling
30-year periods.  The Trinity Study \cite{cooley1998} extended this
analysis across stock/bond allocations.

Guyton and Klinger \cite{guyton2006} introduced dynamic decision rules
(guardrails) that adjust spending based on portfolio performance, finding
that dynamic policies support higher initial withdrawal rates.  Blanchett,
Kowara, and Chen \cite{blanchett2012} developed optimal withdrawal
strategies considering asset allocation and mortality.  Blanchett, Finke,
and Pfau \cite{blanchett2013} examined the impact of low-yield environments
on safe withdrawal rates.

Hamilton \cite{hamilton1989} introduced the Markov regime-switching model
for economic time series.  Ang and Bekaert \cite{ang2002} and Guidolin
and Timmermann \cite{guidolin2007} applied regime-switching to
portfolio allocation.

Glasserman \cite{glasserman2003} provides the standard reference for
Monte Carlo methods in financial engineering, including variance reduction
techniques such as antithetic variates \cite{hammersley1956}.  The
block bootstrap methodology follows Politis and Romano \cite{politis1994}.

The Ornstein--Uhlenbeck process for inflation modeling draws on the
term structure literature originating with Vasicek \cite{vasicek1977}.

% ============================================================
\section{Conclusion}\label{sec:conclusion}

We have presented \textsc{monteplan}, an open-source Python framework
(v0.5.0) for Monte Carlo simulation of multi-account retirement plans
under realistic tax treatment.  The framework addresses a critical gap
in the landscape of financial planning tools: the absence of an
open-source, auditable system that combines parametric return modeling,
progressive tax brackets, multi-account treatment, and dynamic spending
policies in a single coherent engine.

\subsection{Summary of Contributions}

The framework's key technical contributions are:

\begin{enumerate}
  \item \textbf{Vectorized multi-account engine}: A three-dimensional
    tensor representation $(M \times A \times n)$ with a 12-step monthly
    pipeline that eliminates per-path Python loops, achieving throughput
    suitable for interactive web-based exploration;

  \item \textbf{Four return models}: Multivariate normal, Student-$t$
    (fat tails), Markov regime-switching (structural breaks), and
    historical block bootstrap (non-parametric), all sharing a common
    protocol for seamless substitution;

  \item \textbf{Tax-aware decumulation}: US federal progressive brackets
    with preferential LTCG rates, the 3.8\% NIIT surtax, configurable
    state tax overlays, RMD enforcement, and two Roth conversion
    strategies (fixed-amount and fill-bracket);

  \item \textbf{Five dynamic spending policies}: From the simple
    (constant real) to the sophisticated (Guyton--Klinger guardrails),
    enabling exploration of the trade-off between spending stability
    and portfolio survival;

  \item \textbf{Variance reduction}: Antithetic variates providing
    approximately 2$\times$ variance reduction, with formal proof of
    effectiveness for monotone portfolio value functions;

  \item \textbf{Analytics framework}: One-at-a-time sensitivity analysis
    with parallel execution, two-dimensional grid search for interaction
    effects, bisection-based safe withdrawal rate finder, and
    distributional metrics (max drawdown, spending volatility,
    ruin-by-age curves).
\end{enumerate}

\subsection{Design Principles}

The framework's design adheres to four core principles:

\begin{enumerate}
  \item \textbf{Vectorization over explicitness}: All operations across
    Monte Carlo paths use NumPy broadcasting rather than Python loops.
    This enables simulation of 20{,}000 paths over 40 years (480 monthly
    steps) in seconds on commodity hardware;

  \item \textbf{Protocol-based extensibility}: Return models, spending
    policies, and tax models implement Python protocols (structural
    typing), enabling new implementations without modifying the engine;

  \item \textbf{Reproducibility by construction}: Deterministic RNG
    (PCG64DXSM) with user-specified seeds, combined with SHA-256 config
    hashing, ensures every simulation result is exactly replicable and
    traceable to its inputs;

  \item \textbf{Validation at every level}: Golden tests guard against
    regression, table-driven tests verify tax computations against IRS
    publications, and invariant tests ensure mathematical properties
    hold across all inputs.
\end{enumerate}

\subsection{Limitations}

The current version (v0.5.0) has several limitations:

\begin{itemize}
  \item \textbf{Estate planning}: No bequest optimization or estate tax
    modeling;
  \item \textbf{Social Security optimization}: Benefits are modeled as
    exogenous guaranteed income; the claiming age decision is not
    optimized;
  \item \textbf{Dynamic asset allocation}: Only static and linear glide
    path allocations are supported; no market-condition-responsive
    strategies;
  \item \textbf{Healthcare costs}: No age-dependent healthcare expenditure
    modeling (e.g., Medicare Part~B, long-term care);
  \item \textbf{Mortality risk}: The simulation runs to a fixed end age
    rather than incorporating actuarial life tables;
  \item \textbf{Tax code evolution}: Tax brackets are static (2024
    values); no modeling of legislative changes or TCJA sunset;
  \item \textbf{International taxes}: Only US federal and state taxes
    are implemented; no support for other jurisdictions.
\end{itemize}

\subsection{Future Work}

Planned extensions include:

\begin{enumerate}
  \item \textbf{Mortality-weighted metrics}: Integrating actuarial life
    tables (SOA 2012 IAM) to compute mortality-weighted success
    probabilities and expected bequest values;
  \item \textbf{Social Security optimization}: Brute-force or dynamic
    programming search over claiming ages (62--70) for individual
    and spousal benefits;
  \item \textbf{Multi-objective optimization}: Jointly optimizing
    spending level, asset allocation, and Roth conversion strategy
    using Pareto-front methods;
  \item \textbf{Tax-loss harvesting}: Modeling the tax benefit of
    realizing losses in taxable accounts to offset gains;
  \item \textbf{Healthcare cost modeling}: Age-dependent healthcare
    expenditure curves calibrated to CMS data.
\end{enumerate}


% ============================================================
% APPENDICES
% ============================================================
\appendix

% ------------------------------------------------------------
\section{US Federal Tax Brackets (2024)}\label{app:brackets}

\begin{table}[H]
\centering
\caption{2024 ordinary income brackets (single filer).}
\label{tab:brackets}
\begin{tabular}{@{}rr@{}}
\toprule
Taxable Income Up To & Marginal Rate \\
\midrule
\$11,600 & 10\% \\
\$47,150 & 12\% \\
\$100,525 & 22\% \\
\$191,950 & 24\% \\
\$243,725 & 32\% \\
\$609,350 & 35\% \\
Above & 37\% \\
\bottomrule
\end{tabular}
\hfill
\begin{tabular}{@{}rr@{}}
\toprule
Taxable Income Up To & Marginal Rate \\
\midrule
\$23,200 & 10\% \\
\$94,300 & 12\% \\
\$201,050 & 22\% \\
\$383,900 & 24\% \\
\$487,450 & 32\% \\
\$731,200 & 35\% \\
Above & 37\% \\
\bottomrule
\end{tabular}

\smallskip
\footnotesize
Left: Single.  Right: Married Filing Jointly.\\
Standard deduction: \$14,600 (single), \$29,200 (MFJ).\\
Source: IRS Rev.\ Proc.\ 2023-34 \cite{irsrevproc2023}.
\end{table}

\begin{table}[H]
\centering
\caption{2024 long-term capital gains brackets.}
\label{tab:ltcg}
\begin{tabular}{@{}rrr@{}}
\toprule
Filing Status & Threshold & Rate \\
\midrule
\multirow{3}{*}{Single} & Up to \$47,025 & 0\% \\
 & \$47,025--\$518,900 & 15\% \\
 & Above \$518,900 & 20\% \\
\midrule
\multirow{3}{*}{MFJ} & Up to \$94,050 & 0\% \\
 & \$94,050--\$583,750 & 15\% \\
 & Above \$583,750 & 20\% \\
\bottomrule
\end{tabular}
\end{table}

% ------------------------------------------------------------
\section{IRS Uniform Lifetime Table (RMD Divisors)}\label{app:rmd}

\begin{table}[H]
\centering
\caption{IRS Uniform Lifetime Table divisors (selected ages).
Full table covers ages 73--120.}
\label{tab:rmd}
\begin{tabular}{@{}cccccc@{}}
\toprule
Age & Divisor & Age & Divisor & Age & Divisor \\
\midrule
73 & 26.5 & 83 & 17.7 & 93 & 10.1 \\
74 & 25.5 & 84 & 16.8 & 94 & 9.5 \\
75 & 24.6 & 85 & 16.0 & 95 & 8.9 \\
76 & 23.7 & 86 & 15.2 & 96 & 8.4 \\
77 & 22.9 & 87 & 14.4 & 97 & 7.8 \\
78 & 22.0 & 88 & 13.7 & 98 & 7.3 \\
79 & 21.1 & 89 & 12.9 & 99 & 6.8 \\
80 & 20.2 & 90 & 12.2 & 100 & 6.4 \\
81 & 19.4 & 91 & 11.5 & 105 & 4.6 \\
82 & 18.5 & 92 & 10.8 & 110 & 3.5 \\
\bottomrule
\end{tabular}

\smallskip
\footnotesize
Source: IRS Publication 590-B, Table III \cite{irs590b}.
\end{table}

% ------------------------------------------------------------
\section{Complete Parameter Reference}\label{app:params}

\footnotesize
\begin{longtable}{@{}p{3.5cm}p{1.8cm}p{1.8cm}p{5.8cm}@{}}
\caption{Configuration parameter reference.}
\label{tab:params} \\
\toprule
Parameter & Type & Default & Description \\
\midrule
\endfirsthead
\toprule
Parameter & Type & Default & Description \\
\midrule
\endhead
\bottomrule
\endfoot

\multicolumn{4}{@{}l}{\textit{PlanConfig}} \\
\midrule
current\_age & int & --- & Current age (18--100) \\
retirement\_age & int & --- & Retirement age ($>$ current) \\
end\_age & int & --- & Planning horizon (18--120) \\
monthly\_spending & float & --- & Monthly goal (today's \$) \\
monthly\_income & float & 0 & Earned income \\
income\_growth\_rate & float & 0.0 & Real income growth rate \\
income\_end\_age & int$|$None & retire & Age income stops \\
accounts & list & --- & Account configs \\
discrete\_events & list & [] & One-time cash events \\
guaranteed\_income & list & [] & GI streams \\

\midrule
\multicolumn{4}{@{}l}{\textit{MarketAssumptions}} \\
\midrule
expected\_returns & list[float] & --- & Per-asset annual returns \\
annual\_volatilities & list[float] & --- & Per-asset volatility \\
correlation\_matrix & list[list] & --- & $n \times n$ correlation \\
inflation\_mean & float & 0.03 & Long-run inflation \\
inflation\_vol & float & 0.01 & Inflation volatility \\
return\_model & str & mvn & mvn, student\_t, bootstrap, regime\_switching \\
degrees\_of\_freedom & float$|$None & None & Student-$t$ d.o.f.\ ($>2$) \\
bootstrap\_block\_size & int & 12 & Block size (months) \\
expense\_ratio & float & 0.0 & Annual fund fees \\
aum\_fee & float & 0.0 & AUM fee \\
advisory\_fee & float & 0.0 & Advisor fee \\

\midrule
\multicolumn{4}{@{}l}{\textit{PolicyBundle}} \\
\midrule
spending.policy\_type & str & const\_real & Spending policy \\
rebalancing\_strategy & str & calendar & calendar$|$threshold \\
rebalancing\_months & list[int] & [1,7] & Calendar months \\
rebalancing\_threshold & float & 0.05 & Drift threshold \\
withdrawal\_order & list[str] & [tax,trad, roth] & Account priority \\
tax\_model & str & flat & flat$|$us\_federal \\
tax\_rate & float & 0.22 & Flat effective rate \\
filing\_status & str & single & single$|$married\_jointly \\
state\_tax\_rate & float & 0.0 & State flat rate (0--15\%) \\
include\_niit & bool & False & NIIT 3.8\% surtax \\

\midrule
\multicolumn{4}{@{}l}{\textit{RothConversionConfig}} \\
\midrule
enabled & bool & False & Enable conversions \\
strategy & str & fixed\_amt & fixed\_amount, fill\_bracket \\
annual\_amount & float & 0 & Fixed conversion \$ \\
fill\_to\_bracket & float & 0.22 & Target marginal rate \\
start\_age & float & 55 & Conversion start age \\
end\_age & float & 72 & Conversion end age \\

\midrule
\multicolumn{4}{@{}l}{\textit{SimulationConfig}} \\
\midrule
n\_paths & int & 5000 & Monte Carlo paths \\
seed & int & 42 & Random seed \\
antithetic & bool & False & Antithetic variates \\
preset & str$|$None & None & fast$|$balanced$|$deep \\
store\_paths & bool & False & Store full paths \\

\end{longtable}


% ------------------------------------------------------------
\section{Streamlit Application}\label{app:streamlit}

The web application \cite{streamlit2024} provides six pages for
interactive exploration:

\begin{enumerate}
  \item \textbf{Plan Setup}: Demographics, accounts, income, spending,
    guaranteed income streams, discrete events.  Quick-start templates
    (Default, FIRE, Coast FIRE, Conservative Retiree).
  \item \textbf{Portfolio}: Asset allocation, correlation matrix, return
    model selection (MVN, Student-$t$, regime-switching, bootstrap), fees,
    glide path.
  \item \textbf{Run \& Results}: Simulation execution with fan chart,
    spending fan chart, ruin-by-age curve.  SWR finder expander.  CSV
    export.
  \item \textbf{Policies}: Spending policy configuration, rebalancing,
    withdrawal order, tax model, state tax, NIIT, Roth conversions.
  \item \textbf{Compare Scenarios}: Save/load scenarios, overlay fan
    charts, dominance scatter plot.
  \item \textbf{Sensitivity}: OAT tornado chart, 2D heatmap grid.
\end{enumerate}

All simulation results are cached via \texttt{st.cache\_data} keyed by
the configuration SHA-256 hash.


% ============================================================
% BIBLIOGRAPHY
% ============================================================
\bibliographystyle{plainnat}
\bibliography{monteplan}

\end{document}
